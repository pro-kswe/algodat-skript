% !TEX root = ../../../main.tex

\toggletrue{image}
\toggletrue{imagehover}
\chapterimage{ineffective_sorts}
\chapterimagetitle{INEFFECTIVE SORTS}
\chapterimageurl{https://xkcd.com/1185/}
\chapterimagehover{StackSort connects to StackOverflow, searches for 'sort a list', and downloads and runs code snippets until the list is sorted.}

\chapter{Sortieralgorithmen}
\label{chapter-sortieralgorithmen}

\newcommand{\sortieralgorithmenLernziele}{
\protect\begin{todolist}
\item Sie definieren das Sortierproblem für Zahlen.
\item Sie erklären das Prinzip eines Sortieralgorithmus.
\item Sie wenden einen Sortieralgorithmus an.
\item Sie führen die (asymptotische) Laufzeitanalyse für einen Sortieralgorithmus durch.
\end{todolist}
} 

\lernziel{\autoref{chapter-sortieralgorithmen}, \nameref{chapter-sortieralgorithmen}}{\protect\sortieralgorithmenLernziele}

\sortieralgorithmenLernziele

\section{Analyse der asymptotischen Laufzeit:}

\begin{itemize}
\item \textbf{Elementaroperation}:
\end{itemize}

\subsection{Best-Case}

\vspace{3cm}

\subsection{Worst-Case}

\vspace{1.25cm}

\begin{table}[htb]
\renewcommand{\arraystretch}{1.5}
\begin{tabular}{|c|p{2cm}|p{3cm}|}
\hline
Run & Jasskarten ($n=9$) & Allgemein ($n$) \\ \hline
    &            &           \\ \hline
    &            &           \\ \hline
    &            &           \\ \hline
    &            &           \\ \hline
    &            &           \\ \hline
    &            &           \\ \hline
    &            &           \\ \hline
    &            &           \\ \hline
\end{tabular}
\end{table}

\newpage

\section{Analyse der asymptotischen Laufzeit:}

\begin{itemize}
\item \textbf{Elementaroperation}:
\end{itemize}

\subsection{Best-Case}

\vspace{5cm}

\subsection{Worst-Case}

\vspace{1.25cm}

\begin{table}[htb]
\renewcommand{\arraystretch}{1.5}
\begin{tabular}{|c|p{2cm}|p{3cm}|}
\hline
Run & Jasskarten ($n=9$) & Allgemein ($n$) \\ \hline
    &            &           \\ \hline
    &            &           \\ \hline
    &            &           \\ \hline
    &            &           \\ \hline
    &            &           \\ \hline
    &            &           \\ \hline
    &            &           \\ \hline
    &            &           \\ \hline
\end{tabular}
\end{table}

\newpage

\section{Analyse der asymptotischen Laufzeit:}

\begin{itemize}
\item \textbf{Elementaroperation}:
\end{itemize}

\subsection{Best-Case}

\vspace{3cm}

\subsection{Worst-Case}

\newpage

\section{Analyse der asymptotischen Laufzeit:}

\begin{itemize}
\item \textbf{Elementaroperation}:
\end{itemize}

\vspace{5cm}

\subsection{1. Run}

\begin{figure}[htb]
\centering
\begin{tikzpicture}[every node/.style={minimum size=0.75cm, inner sep=0pt}, level/.style={sibling distance=5.5cm/#1}, edge from parent/.style={draw, -latex}]
\node [circle, draw] (v0) {}
	child {
		node [circle, draw] (v1) {}
		child {
			node [circle, draw] (v2) {}
			child {
				node [circle ,draw, fill=green!25] (v3) {$13$}
			}
			child {
				node [circle, draw, fill=green!25] (v4) {$9$}
			}
		}
		child {
			node [circle, draw] (v5) {}
			child {
				node [circle, draw, fill=green!25] (v6) {$10$}
			}
			child {
				node [circle, draw, fill=green!25] (v7) {$7$}
			}
		}
   	}
	child {
		node [circle, draw] (v8) {}
		child {
			node [circle, draw] (v9) {}
			child {
				node [circle, draw, fill=green!25] (v10) {$6$}
			}
			child {
				node [circle, draw, fill=green!25] (v11) {$11$}
			}
		}
		child {
			node [circle, draw] (v12) {}
			child {
				node [circle, draw, fill=green!25] (v13) {$8$}
			}
			child {
				node [circle, draw, fill=green!25] (v14) {$12$}
			}
		}
	};
\node[right=of v0, font=\bfseries] (l0) {Level 0};
\node[right=of v8, font=\bfseries] (l1) {Level 1};
\node[right=of v12, font=\bfseries] (l2) {Level 2};
\node[right=of v14, font=\bfseries] (l3) {Level 3};
\end{tikzpicture}
\end{figure}

\newpage

\subsection{Restliche Runs:}

\begin{figure}[htb]
\centering
\begin{tikzpicture}[every node/.style={minimum size=0.75cm, inner sep=0pt}, level/.style={sibling distance=5.5cm/#1}, edge from parent/.style={draw, -latex}]
\node [circle, draw] (v0) {}
	child {
		node [circle, draw] (v1) {}
		child {
			node [circle, draw] (v2) {}
			child {
				node [circle ,draw, fill=green!25] (v3) {$13$}
			}
			child {
				node [circle, draw, fill=green!25] (v4) {$9$}
			}
		}
		child {
			node [circle, draw] (v5) {}
			child {
				node [circle, draw, fill=green!25] (v6) {$10$}
			}
			child {
				node [circle, draw, fill=green!25] (v7) {$7$}
			}
		}
   	}
	child {
		node [circle, draw] (v8) {}
		child {
			node [circle, draw] (v9) {}
			child {
				node [circle, draw, fill=green!25] (v10) {$6$}
			}
			child {
				node [circle, draw, fill=green!25] (v11) {$11$}
			}
		}
		child {
			node [circle, draw] (v12) {}
			child {
				node [circle, draw, fill=green!25] (v13) {$8$}
			}
			child {
				node [circle, draw, fill=green!25] (v14) {$12$}
			}
		}
	};
\node[right=of v0, font=\bfseries] (l0) {Level 0};
\node[right=of v8, font=\bfseries] (l1) {Level 1};
\node[right=of v12, font=\bfseries] (l2) {Level 2};
\node[right=of v14, font=\bfseries] (l3) {Level 3};
\end{tikzpicture}
\end{figure}

\newpage

\subsection{Worst Case Laufzeit}

\vspace{3cm}


\subimport{bubblesort}{bubblesort}

\subimport{selectionsort}{selectionsort}