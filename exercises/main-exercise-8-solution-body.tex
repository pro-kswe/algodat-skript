% ------------------------------------------------------------------------
% file `main-exercise-8-solution-body.tex'
%   in folder `exercises/'
%
%     solution of type `exercise' with id `8'
%
% generated by the `solution' environment of the
%   `xsim' package v0.21 (2022/02/12)
% from source `main' on 2023/05/25 on line 16
% ------------------------------------------------------------------------
\begin{minipage}{\linewidth}
\begin{lstlisting}[language=pseudocode, caption={Algorithmus für das Problem \protect\autoref{problem-min-median-max-n-zahlen}}, label={lst-algo-min-median-max-n-zahlen}]
input: Ein Array A mit n Zahlen.
# In einem aufsteigend sortieren Array ist das Minimum
# die erste Zahl und das Maximum immer die letzte Zahl.
minimum $\gets$ A[0]
maximum $\gets$ A[n - 1]

# Prüfen, ob es eine ungerade Anzahl von Zahlen ist
# oder nicht
if n % 2 = 1 {
 # Bei einer ungeraden Anzahl von Zahlen gibt es eine
 # eindeutige Mitte. // ist die ganzzahlige Division
 # (Bsp.: 7 // 2 ergibt 3)
 indexMitte $\gets$ n // 2
 median = A[indexMitte]
}
else {
 # Bei einer geraden Anzahl von Zahlen gibt es keine
 # eindeutige Mitte. Wir berechnen den Durchschnitt der
 # beiden mittleren Zahlen.
 index_1 = n // 2
 index_2 = (n // 2) - 1
 median = (A[index_1] + A[index_2]) / 2
}
output: minimum, median, maximum
\end{lstlisting}
\end{minipage}
