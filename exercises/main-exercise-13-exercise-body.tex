% ------------------------------------------------------------------------
% file `main-exercise-13-exercise-body.tex'
%   in folder `exercises/'
%
%     exercise of type `exercise' with id `13'
%
% generated by the `exercise' environment of the
%   `xsim' package v0.21 (2022/02/12)
% from source `main' on 2023/05/25 on line 296
% ------------------------------------------------------------------------
Entwerfen Sie einen Algorithmus in Pseudocode für folgendes Problem. Tipp: Verwenden Sie den Pseudocode eines Suchalgorithmus als Grundlage.

\begin{problem}[Positionssuche-n-Zahlen]\label{problem-positionssuche-n-zahlen}
Es ist ein Array $A$ mit $n$ verschiedenen Zahlen und eine Zahl \lstinline[language=pseudocode]{k} gegeben. Die Positionssuche soll ermitteln, an welcher \textbf{Position} (Ausgabe soll der Index sein) die gegebene Zahl \lstinline[language=pseudocode]{k} vorhanden ist. Falls die Zahl \textbf{nicht} vorhanden ist, dann soll als Ausgabe für die Position \lstinline[language=pseudocode]{-1} verwendet werden.
\end{problem}

\fillwithgrid	{\stretch{1}}

